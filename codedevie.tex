\documentclass{aediroum}

\title{Code de vie}
\date{6 novembre 2024\\[.5em]\emph{Adoption provisoire en réunion du C.E.}}

\begin{document}
\maketitle

\section{Définitions}

\begin{description}
    \item[AÉDIROUM] Association Étudiante du Département d’Informatique et de Recherche Opérationnelle de l’Université de Montréal
    \item[Charte] La Charte de l’AÉDIROUM
    \item[C.R.] Comité de représentation de l’AÉDIROUM, tel que défini à l’article~4 de la Charte
    \item[C.E.] Conseil exécutif de l’AÉDIROUM, tel que défini à l’article~5 de la Charte
    \item[COUSSIN] Comité Organisant l’Univers Social et Sportif INformatique, tel que défini par le Règlement de la vie étudiante
    \item[CÉAVVH] Comité d’Écoute et d’Assistance aux Victimes de Violence et de Harcèlement
    \item[Comportement problématique] Tout cas de violence verbale ou physique (sauf en cas de légitime défense), d’incitation à la violence, d’incivilité, de menace, ou de harcèlement verbal ou physique
    \item[Récidive] Répétition d’un Comportement problématique de la même nature par la même personne après qu’elle ait été sanctionnée par le CÉAVVH, sans égard au temps écoulé entre les deux situations ni au contexte dans lequel les situations se sont produites
\end{description}

\section{Code de vie}

L’AÉDIROUM garantit à chaque membre le droit à un environnement social et d’étude sain et exempt de toute forme de violence, d’incivilité, de harcèlement, de discrimination et de racisme.

L’AÉDIROUM met en œuvre des interventions proportionnelles à la complexité des Comportements problématiques et aux enjeux qu’ils soulèvent.

Les dispositions du présent document s’appliquent à tout espace ou activité sous la responsabilité de l’AÉDIROUM, incluant sans s’y restreindre les locaux de l’AÉDIROUM, les événements organisés par l’AÉDIROUM et les espaces numériques gérés par l’AÉDIROUM.

\section{Comité}

Ce document est constitutif du CÉAVVH, un comité permanent ayant pour mission de gérer les plaintes en lien avec le code de vie selon les procédures définies ci-après. Ce comité est régi par l’article 6 de la Charte.

\subsection{Personnes membres}

Le comité est composé d’au moins deux (2) et d’au plus six (6) personnes membres. Le comité doit, dans la mesure du possible, être paritaire et inclure au moins une personne élue du COUSSIN et une du C.R.

Les personnes membres du CÉAVVH sont nommées par le C.E. lors de sa première réunion à la session d’automne. Le C.E. peut, en tout temps, décider de retirer ou remplacer des membres ou de combler les postes vacants. En vertu de la Charte, la présidence de l’AÉDIROUM est membre d’office du comité.

\subsection{Missions}

Le CÉAVVH agit de manière à:

\begin{enumerate}
    \item Donner l’exemple quant au respect du code de vie.
    \item Protéger les victimes et les personnes témoignantes, en particulier leur anonymat lorsque celui-ci est demandé par l’une des parties, ou lorsque celui-ci est jugé nécessaire par le comité.
    \item Respecter la crédibilité de la parole des victimes.
    \item Offrir l’opportunité aux personnes accusées de se défendre.
    \item Fonder toute décision sur des preuves suffisantes et selon un principe de proportionnalité et d’équité des sanctions.
    \item Respecter la stricte confidentialité des informations dont il a connaissance.
\end{enumerate}

\section{Procédure}

\subsection{Réception des plaintes}

Le CÉAVVH a le devoir de recevoir toute plainte relative au code de vie formulée par une personne membre de l’AÉDIROUM ou visant une personne membre de l’AÉDIROUM.

Dès la réception d’une plainte, une personne membre du comité doit en faire part au reste du comité.

Le comité a la liberté de regrouper plusieurs plaintes ensemble s’il estime qu’elles doivent être traitées en bloc.

\subsection{Décision du comité}

Suite à la réception d’une plainte, le CÉAVVH peut décider d’imposer toute sanction qu’il juge appropriée, selon l’échelle définie dans la section suivante.

Toute décision du comité doit être prise à l’unanimité. En cas de désaccord, aucune sanction n’est appliquée.

Les membres du C.E. et du COUSSIN sont responsables de l’application de toute décision prise par le comité.

\subsubsection{Échelle de sanctions}

\begin{enumerate}
    \item Avertissement de la part de l’association, en personne et par écrit, rappelant notamment à la personne avertie le Comportement problématique qui lui est reproché et les attentes de l’association vis à vis de la personne, incluant les prochaines étapes en cas de récidive.
    \item Suspension de la possibilité de fréquenter les espaces communs de l’association, de communiquer dans les espaces numériques de l’association, et de participer aux activités ou évènements pour une durée de huit (8) semaines. Lorsque la personne visée est élue de l’AÉDIROUM, sa destitution est recommandée auprès du C.E.
    \item Suspension de la possibilité de fréquenter les espaces communs de l’association, de communiquer dans les espaces numériques de l’association, et de participer aux activités ou évènements pour une durée de seize (16) semaines. Lorsque la personne visée est élue de l’AÉDIROUM, sa destitution est recommandée auprès du C.E.
    \item Exclusion totale et permanente des espaces communs de l’association, des espaces numériques de l’association, et des activités ou événements à partir de la date fixée. Lorsque la personne visée est élue de l’AÉDIROUM, sa destitution est recommandée auprès du C.E.
\end{enumerate}

En cas de récidive, le comité doit incrémenter l’échelon de la sanction relativement à sa décision précédente.

Lorsqu’une sanction s’applique pour une certaine durée, le comité décide de la ou les périodes pendant lesquelles la sanction a cours, de sorte que la durée totale des périodes corresponde à celle prescrite dans le présent document.

\subsection{Décision urgente}

Toute personne responsable d’un événement peut décider d’exclure une personne de dudit événement si elle estime qu’elle contrevient au code de vie pendant l’événement. La personne responsable doit obligatoirement faire un rapport de la situation au CÉAVVH dans un délai de cinq (5) jours ouvrables.

\subsection{Divulgation d’informations}

En cas de nécessité absolue, dans l’intérêt et avec le consentement des victimes, le CÉAVVH peut décider de divulguer certaines informations relatives à une plainte à des instances internes ou externes, incluant sans s’y restreindre des instances de l’université.

\subsection{Conservation des plaintes}

Sans contrevenir à son principe de confidentialité, le CÉAVVH conserve, sans limite de temps, l’historique des plaintes reçues et des décisions prises.

\end{document}
