\documentclass[twoside]{article}
\usepackage[french]{babel}
\usepackage[utf8]{inputenc}
\usepackage[T1]{fontenc}
\usepackage{marginnote}

\newenvironment{instance}
{\flushright\itshape\scriptsize}

\title{Cahier de positions}
\author{AÉDIROUM}
\date{À jour du \today}

\begin{document}

\maketitle

Le présent document regroupe l'ensemble des positions de l'AÉDIROUM. Il est utilisé par toute personne faisant de la représentation au nom de l'association étudiante. Ce document doit être mis à jour régulièrement afin de bien refléter les positions des membres de l'AÉDIROUM.

Ce document a été créé à la session d'hiver 2011 par l'exécutif en place afin de recenser les opinions de ses membres, abrogé en 2016, puis reconstitué en 2021. Les positions sont organisées en deux sections selon leur champ d'application.

\section{Affaires externes}
\begin{enumerate}
	\item \marginnote{Frais de scolarité} Que l'AÉDIROUM travaille afin que le système d'éducation supérieure québécois:
		\begin{itemize}
			\item Soit également accessible indépendamment de la situation financière des étudiants et de leurs proches;
			\item Permette à un maximum d'étudiants compétents de compléter leurs projets d'études et d'emploi.
		\end{itemize}
		\begin{instance}
			Adopté: [AG-02/11/2011]
		\end{instance}

	\item Que l'AÉDIROUM s'oppose à toute hausse des frais de scolarité, en perspective de la gratuité scolaire.
		\begin{instance}
			Adopté: [AG-07/04/2011], Modifié: [AG-10/02/2012], Modifié: [AG-16/03/2012]
		\end{instance}

	\item \marginnote{Environnement} Que l'AÉDIROUM mette de l'avant et défende les initiatives environnementales.
		\begin{instance}
			Adopté: [AG-25/01/2011]
		\end{instance}

	\item \marginnote{Transport en commun} Que l'AÉDIROUM appuie des démarches en vue d'avoir un tarif réduit pour tous les étudiants à temps plein.
		\begin{instance}
			Adopté: [AG-07/04/2011], Modifié: [AG-14/09/2011]
		\end{instance}

	\item \marginnote{Accès à l'information} Que la FAÉCUM facilite l'accès à ses documents, notamment en permettant de rendre public ses documents d'intérêt général, dont ses règlements généraux et sa politique d'accès à l'information.
		\begin{instance}
			Adopté: [AG-14/09/2011]
		\end{instance}

	\item \marginnote{Mode de scrutin} Que l'AÉDIROUM soutienne la cause du SENSÉ dans ses démarches pour un nouveau mode de scrutin à mi-chemin entre la représentation circonscriptions et le vote proportionnel mixte. Ce mode permet aux partis politiques émergents d'être moins pénalisés par le système de circonscriptions et surtout de donner un pourcentage de pouvoir à chaque parti qui correspond au pourcentage réel de la population derrière chaque parti.
		\begin{instance}
			Adopté: [AG-04/04/2018]
		\end{instance}

	\item \marginnote{Aide financière} Que l'AÉDIROUM soit pour un régime d'aide financière adéquat ayant pour but d'éliminer l'endettement étudiant et d'assurer la satisfaction des besoins fondamentaux.
		\begin{instance}
			Adopté: [AG-14/02/2014]
		\end{instance}

	\item \marginnote{Entreprises privées} Que l'AÉDIROUM soit pour un réseau d'éducation public libre de toute ingérence de l'entreprise privée au niveau de la gouvernance des institutions publiques, et que l'AÉDIROUM condamne la culture de sous-traitance abusive dans les institutions publiques d'éducation.
		\begin{instance}
			Adopté: [AG-14/02/2014]
		\end{instance}

	\item Que l'AÉDIROUM soit contre toute forme de mondialisation qui entérine la prédominance du profit sur le bien-être de la population.
		\begin{instance}
			Adopté: [AG-14/02/2014]
		\end{instance}

	\item \marginnote{Inclusivité} Que l'AÉDIROUM dénonce toute forme de discrimination ciblant l'identité ou l'expression de genre.
		\begin{instance}
			Adopté: [AG-14/02/2014]
		\end{instance}

	\item Que l'AÉDIROUM appuie le Groupe d'action trans* de l'Université de Montréal dans ses démarches pour une université inclusive.
		\begin{instance}
			Adopté: [AG-14/02/2014]
		\end{instance}

	\item Que l'AÉDIROUM invite l'Université de Montréal~:
		\begin{enumerate}
			\item à modifier sa Politique contre le harcèlement afin d'interdire explicitement la discrimination envers l'identité ou l'expression de genre, et à promouvoir l'acceptation des personnes trans*~;
			\item à faciliter les démarches permettant de concilier l'identité de genre des étudiants et leur statut administratif, notamment en permettant aux étudiants d'utiliser leur nom, leur formule d'appel et leur mention de sexe préférés sur leur carte étudiante, sur leur horaire, sur les listes de classe, sur l'environnement StudiUM, dans leur courriel institutionnel, et dans tout autre contexte où il est possible de le faire~;
			\item à créer des espaces plus accessibles pour les personnes trans*, notamment en incluant des toilettes neutres sur ses plans de construction ou de rénovation futurs ou en affichant les toilettes à une place comme neutres~;
			\item à faciliter, dans la mesure du possible, l'accès aux soins de santé que requièrent les personnes trans* à la clinique universitaire, et à éduquer le personnel soignant pour qu'il sache bien gérer les personnes trans*~;
			\item à diffuser, tant aux étudiant-e-s qu'aux employé-e-s, des informations sur l'identité et l'expression de genre et sur les politiques de l'Université visant les personnes trans*.
		\end{enumerate}
		\begin{instance}
			Adopté: [AG-14/02/2014]
		\end{instance}
\end{enumerate}

\section{Affaires internes}
\begin{enumerate}
	\item \marginnote{Rétroaction de mi-session} Que l'AÉDIROUM s'assure de la tenue d'une rétroaction de mi-session toutes les sessions d'automne et d'hiver.
		\begin{instance}
			Adopté: [AG-25/01/2011]
		\end{instance}

	\item \marginnote{Logiciels et formats} Que l'AÉDIROUM préconise l'utilisation de logiciels libres et de formats ouverts.
		\begin{instance}
			Adopté: [AG-25/01/2011], Modifié: [AG-7/04/2011], Modifié: [AG-14/09/2011]
		\end{instance}

	\item \marginnote{Politique linguistique} Que l'AÉDIROUM s'assure que tous les cours et le matériel d'apprentissage soient disponibles en français au premier cycle.
		\begin{instance}
			Adopté: [AG-25/01/2011]
		\end{instance}

	\item Que l'AÉDIROUM défende le choix des étudiants en ce qui a trait au choix de la langue d'écriture de leur thèse de doctorat ou de leur mémoire de maîtrise.
		\begin{instance}
			Adopté: [AG-25/01/2011], Modifié: [AG-14/09/2011], Modifié: [AG-10/02/2012]
		\end{instance}

	\item \marginnote{Identité} Que l'AÉDIROUM reconnaisse l'identité de genre déclarée de tout-e-s ses membres dans le cadre de toutes ses activités, peu importe leur statut légal, administratif, chirurgical ou autre.
		\begin{instance}
			Adopté: [AG-14/02/2014]
		\end{instance}

	\item Que l'AÉDIROUM se positionne au niveau départemental pour qu'on puisse utiliser son nom usuel sur les travaux et les examens.
		\begin{instance}
			Adopté: [AG-14/02/2014]
		\end{instance}
\end{enumerate}

\end{document}
