\documentclass{aediroum}

\newcommand{\article}[1]{article \ref{#1}}

\title{Règlement sur l'affiliation à la FAÉCUM}
\date{Mars 2022}

\begin{document}
\maketitle

\section{Préambule}
Ce Règlement a été formé avec les instructions émises dans la Charte afin de mieux baliser et clarifier les relations de l'AÉDIROUM à la FAÉCUM. Le déplacement comme annexe permet aussi une plus grande flexibilité pour s'adapter à leurs fonctions et de notre implication.

\section{Dispositions}

\subsection{Définitions}
\begin{description}
	\item[AÉDIROUM] Association des Étudiants du Département d'Informatique et de Recherche Opérationnelle de l'Université de Montréal
	\item[Charte] La Charte de l'AÉDIROUM
	\item[CA] Conseil d'administration de l'AÉDIROUM, tel que défini à l'article 3.3 de la Charte de l'AÉDIROUM
	\item[CE] Conseil exécutif de l'AÉDIROUM, tel que défini à l'article 3.3 de la Charte de l'AÉDIROUM
	\item[FAÉCUM] Fédération des associations étudiantes du campus de l'Université de Montréal
	\item[Règlement] Règlement sur l'affiliation à la FAÉCUM
	\item[Intervention] Parole lors d'un débat et vote
\end{description}

\subsection{Modification du guide}
L'assemblée générale des Membres de l'AÉDIROUM peut voter à majorité simple toutes modifications au règlement et à ses annexes, et ce sans que ces modifications figurent à la convocation.

Le CA est habilité à modifier le Règlement par un vote du deux tiers (arrondi vers le haut) des membres du CA lors d'une séance dudit conseil et est contraint par le présent règlement.

Toute modification aux annexes par le CA doit être entérinée à majorité simple lors de l'Assemblée Générale suivant l'adoption de la modification.

\section{Participation aux instances et sous-instances}\label{sec:hierarchie-delegues-faecum}

Tout Intervention des réprésentant de l'AÉDIROUM aux instances et sous-instances de la FAÉCUM est doit respecter la section 7 de la Charte.

\subsection{Délégués d'office aux instances et sous-instances}\label{sec:delegues-doffice-instances}
\begin{description}
\item[Congrès] Vice-président externe
\item[Conseil Central (CC)] Vice-président externe
\item[Conseil des affaires académiques (CAA)] Vice-président interne, il devrait aussi avoir un Membre du 1\textsuperscript{er} cycle.
\item[Conseil des études supérieures (CES)] Vice-président interne, il devrait aussi avoir un Membre de 2\textsuperscript{e} ou 3\textsuperscript{e} cycle.
\item[Conseil des affaires socio-politiques (CASP)] Vice-président externe
\item[Conseil à la Vie Étudiante (CVE)] Coordonnateur à la vie étudiante
\end{description}

\subsection{Hiérarchie de délégation}\label{sec:hierarchie-delegues-faecum}

Le représentant officiel de l'AÉDIROUM à une instance ou sous-instance est la personne présente à celle-ci qui est la plus haute dans la hiérarchie suivante. Le délégué d'office à l'instance ou la sous-instance est la personne spécifiée dans l'\article{sec:delegues-doffice-instances}. Si la fonction plus haute dans la hiérarchie ne peut se présenter à l'instance, il revient au prochain point dans la hiérarchie de remplir ce rôle récursivement.

S'il peut avoir plusieurs réprésentants votants, les délégué·e·s sont déterminé·e·s par cette ordre.

\begin{enumerate}\setcounter{enumi}{-1}
\item délégué·e d'office,
\item adjoint·e au délégué·e d'office,
\item président,
\item vice-président externe,
\item vice-président interne,
\item secrétaire,
\item trésorier,
\item tout membre du Conseil d'Administration,
\item tout Membre de l'AÉDIROUM, sous l'approbation d'un membre du CE.
\end{enumerate}

Les Membres peuvent toutefois accompager la délégation aux réunions sans l'approbation du CA. Ils ne peuvent pas se présenter au nom de l'AÉDIROUM.

\end{document}
