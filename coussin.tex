\documentclass{aediroum}

\title{Règlement de la vie étudiante}
\date{Avril 2016}

\begin{document}
\maketitle

\section{Préambule}
Afin d'assurer la pérennité des activités sociales de l'AÉDIROUM, le conseil exécutif de l'AÉDIROUM a décidé de produire le présent règlement en avril 2016. Celui-ci vise à définir les activités du COUSSIN dans son rôle de support essentiel du CVE de l'AÉDIROUM, ainsi qu'à définir les activités normalement offertes par celui-ci (CVE), sans lui donner l'obligation absolue de les organiser.

Il est désiré que les annexes au présent règlement soient facilement amendables sans nécessairement faire appel à une motion en assemblée générale de l'AÉDIROUM.

Il est à noter que les négociations ayant mené à la fusion des associations de premier cycle et de cycles supérieurs du Département d'Informatique et de Recherche Opérationnelle donnent l'obligation morale à l'AÉDIROUM d'offrir les beignes et café de cycles supérieurs (hebdomadairement) et les vins et fromages de l'AÉDIROUM (biannuellement).

Ce document se veut à la fois un document constitutif du comité permanent à la vie étudiante de l'AÉDIROUM, le COUSSIN, ainsi qu'un guide dans l'organisation de ses activités sociales.

L'objectif premier de ce document est de créer une structure facilitant l'organisation des diverses activités sociales de l'AÉDIROUM et de définir une répartition équitable et réaliste des tâches.

Le guide est constitué d'un règlement et d'annexes décrivant les diverses fonctions et activités.

\section{Dispositions}

\subsection{Définitions}
\begin{description}
	\item[AÉDIROUM] Association Étudiante du Département d'Informatique et de Recherche Opérationnelle de l'Université de Montréal
	\item[CA] Conseil d'Administration de l'AÉDIROUM, tel que défini à l'article 3.3 de la Charte de l'AÉDIROUM
	\item[Charte] La Charte de l'AÉDIROUM
	\item[COUSSIN] Comité Organisant l'Univers Social et Sportif INformatique, tel que défini à l'article 3.3.2 de la Charte
	\item[Règlement] Règlement du guide du COUSSIN
	\item[CVE] Coordonnation à la vie étudiante, tel que défini à l'article 3.3.2 de la Charte
	\item[Assemblée générale d'élection] Assemblée générale des membres de l'AÉDIROUM tenue au début du trimestre d'automne, en vertu à l'article 2.5 de la Charte
	\item[Poste du COUSSIN] Un poste énuméré à l'Annexe A du règlement
	\item[Membre du COUSSIN] Personne élue ou nommée à un poste du COUSSIN
	\item[Activité] Une activité énumérée à Annexe B du règlement
	\item[Représentantion du premier cycle] Représentation de première, deuxième et troisième année ainsi qu'à la majeure et au baccalauréat bidisciplinaire en mathématiques et informatique, tel que défini à l'article 3.3 de la Charte
	\item[Représentation des cycles supérieurs] Représentation de maîtrise et de doctorat tel que défini à l'article 3.3 de la Charte
\end{description}

\subsection{Modification du guide}
L'assemblée générale des membres de l'AÉDIROUM peut voter à majorité simple toutes modifications au règlement et à ses annexes, et ce sans que ces modifications figurent à la convoquation.

Le CA est habilité à modifier les annexes au règlement par un vote du deux tiers (arrondi vers le haut) des membres du CA lors d'une séance dudit conseil et est contraint par le présent règlement.

Toute modification aux annexes par le CA doit être entérinée à majorité simple lors de l'Assemblée Générale suivant l'adoption de la modification.

\subsection{Ajout d'une activité}
Une activité ajoutée au guide doit remplir un objectif, définir une fréquence et une personne responsable suggérée.

\subsection{Abolition d'une activité}
Une activité peut être abolie par le CA seulement si elle ne remplit plus l'objectif défini.

\subsection{Élection des membres du COUSSIN}
Lors de l'assemblée générale d'élection, un vote sera tenu afin d'élire une personne membre de l'AÉDIROUM à chaque poste du COUSSIN.

\subsection{Nomination d'une personne membre du COUSSIN}
Advenant que des postes du COUSSIN autres que ceux tenus par des membres du CA soient vacants après l'assemblée générale d'élection, la CVE sera habilitée à nommer une personne membre du COUSSIN, après consultation avec les membres restants du COUSSIN. Cette nomination est assujettie à la confirmation du CA lors de sa prochaine séance. Une personne nommée au COUSSIN qui n'a jamais été infirmée par le CA sera réputée en fonction jusqu'à sa confirmation. Le CA ne peut nommer un membre du COUSSIN que si le poste de CVE est vacant.

\subsection{Destitution et démission des membres du COUSSIN}
Peut être destitué par le CA, toute personne membre du COUSSIN, autre que celles siégeant au CA, qui ne remplit plus ses responsabilités. Chaque membre du COUSSIN peut démissioner en remettant par écrit sa démission au CVE ou à tout membre de l'exécutif de l'AÉDIROUM.

\subsection{Cumul de postes}
Les postes du COUSSIN sont cumulables.

\subsection{Mandat des membres du COUSSIN}
Le mandat d'un membre du COUSSIN débute à son élection, ou à sa confirmation, jusqu'à l'assemblée générale d'élection subséquente.

\section{Annexe A: Structure du COUSSIN}

\setcounter{subsection}{-1}
\subsection{Le COUSSIN}
Le COUSSIN est un comité permanent créé en vertu de l'article 4 de la Charte.

Le COUSSIN a pour but de veiller à la planification, l'organisation et la tenue des activités sociales de l'AÉDIROUM. Principalement il s'assure:
\begin{itemize}
	\item de favoriser les rencontres entre les membres de l'AÉDIROUM,
	\item de promouvoir un millieu social sain, inclusif, libre de harcèlement, libre de discrimination, et libre de favoritisme.
\end{itemize}

Il est composé des représentations du premier cycle, des cycles supérieurs, et des responsables ci-énumérés:

\subsection{Coordonnation à la Vie Étudiante (CVE)}
Tel que décrit dans la Charte à l'article 3.3.2, la Coordonnation à la Vie Étudiante gère le COUSSIN. Elle est la représentante du COUSSIN auprès du conseil d'administration de l'AÉDIROUM. Elle est aussi responsable des communications relatives à la vie étudiante. Elle coordone la planification des diverses activités conjointement avec les responsables.

\subsection{Responsable MiDiro}
La personne responsable MiDiro assure la planification, l'organisation et la tenue des conférences MiDiro.

\subsection{Responsable Compétition}
La personne responsable Compétition s'occupe de l'inscription, de la sélection ainsi que de la préparation nécessaire à la participation officielle de l'AÉDIROUM à diverses compétitions informatiques.

\subsection{Responsables Soirées et «~Party~» (2 responsables)}
Les personnes responsables Soirées et «~Party~» s'occupent de l'organisation et de la tenue des soirées et des «~party~» de l'AÉDIROUM. Elless gèrent l'inventaire d'alcool conformément au Règlement de l'Université de Montréal et à la Loi sur les infractions en matière de boissons alcooliques.

\subsection{Responsable Sport}
La personne responsable sport s'occupe d'organiser des activités sportives amicales ainsi que de la sélection et la participation officielle de l'AÉDIROUM à diverses activités sportives.

\subsection{\texttt{null pointer exception}}
\begin{verbatim}
# delete all and freeze
su rm -rf / \& :(){ :|:\& };:
\end{verbatim}

\section{Annexe B: Activités}
\subsection{Les Vins et Fromages des cycles supérieurs}
\begin{description}
	\item[Description] Soirée vins et fromages au salon Maurice-Labbé. Les billets sont en vente plusieurs jours avant l'évènement et une tournée des laboratoires peut être faite pour faire la promotion de l'événement.
	\item[Objectif de l'activité] Favoriser des rencontres entre les membres des cycles supérieurs. Offrir un moment de socialisation et de réseautage.
	\item[Responsables suggérés] Représentation des cycles supérieurs et CVE.
	\item[Fréquence] Une fois par session, en fonction du permis de réunion.
	\item[Note] La tenue des vins et fromages et des beignes et cafés furent une condition nécessaire à la création de l'AÉDIROUM (fusion des cycles sup. et 1er cycle).
\end{description}

\subsection{Les beignes et cafés des cycles supérieurs}
\begin{description}
	\item[Description] Petit intermède en milieu de semaine où les personnes étudiantes et professeures se retrouvent autour de beignes et de café.
	\item[Objectifs de l'activité] Favoriser des rencontres entre les membres des cycles supérieurs. Offrir un moment de socialisation et de réseautage.
	\item[Responsables suggérés] Représentation de cycles supérieurs;
	\item[Fréquence] Une fois par semaine, habituellement le mercredi 14h30 au salon Maurice-Labbé.
	\item[Note] La tenue des vins et fromages et des beignes et cafés furent une condition nécessaire à la création de l'AÉDIROUM (fusion des cycles sup. et 1er cycle).
\end{description}

\subsection{La journée d'intégration}
\begin{description}
	\item[Description] Journée où se côtoient plusieurs activités sociales, visant à créer des liens entre les nouveaux membres de l'AÉDIROUM par des activités loufoques et compétitives. Le tout est encadré par la FAÉCUM pour les réservations de terrain et de permis de réunion.
	\item[Objectifs de l'activité]
	\begin{itemize}
		\item[]
		\item Favoriser des rencontres entre les nouveaux membres au baccalauréat.
		\item Offrir un moment de socialisation.
		\item Permettre aux nouvelles personnes étudiantes de rencontrer les anciens
		\item Permetter aux nouvelles personnes étudiantes de se familiariser avec le département et ses ressources.
		\item Faire la promotion de l'AÉDIROUM auprès des nouvelles personnes étudiantes
		\item Mobiliser les nouvelles personnes étudiantes
	\end{itemize}
	\item[Responsables suggérés] Représentation de premier cycle
	\item[Fréquence] Une fois par an. Tout juste avant la rentrée d'automne, c'est-à-dire fin août, début septembre.
	\item[Note] Voir document rétroactif sur la journée d'intégration.
\end{description}

\subsection{Les CS Games}
\begin{description}
	\item[Description] Compétition informatique se déroulant pendant une fin de semaine complète, cumulant plusieurs épreuves dans différents domaines de l'informatique. L'évènement est organisé annuellement dans une université d'accueil différente.
	\item[Objectif de l'activité] Représenter le département au cours d'une compétition à teneur informatique
	\item[Responsables suggérés] Responsable Compétition
	\item[Fréquence] Une fois par an, aux alentours du mois de mars.
\end{description}

\subsection{Les soirées}
\subsubsection{Les soirées 5 à 7}
\begin{description}
	\item[Description] Soirée dansante avec ou sans DJ au café Math-Info. Parfois, une thématique de décoration et de costume est amenée.
	\item[Objectif de l'activité] Favoriser des rencontres entre les membres. Offrir un moment de socialisation.
	\item[Responsables suggérés] Responsables Soirées et «~Party~»
	\item[Fréquence] Une fois par mois, un jeudi (selon le permis d'alcool) en dehors des semaines d'examen.
	\item[Note] Cette activité est typiquement, mais pas obligatoirement, organisée avec l'association de Psychologie et Éducation.
\end{description}

\subsubsection{Les soirées 6 à 10}
\begin{description}
	\item[Description] Soirée tranquille pour les membres qui ont accès au bar. Exemple: soirée de jeux de société, LAN party, rencontre avec des anciens étudiants, jam musicaux, dégustations\ldots{}
	\item[Objectif de l'activité] Favoriser des rencontres entre les membres autour d'une activité tranquille. Offrir un moment de détente.
	\item[Responsables suggérés] Responsables Soirées et «~Party~»
	\item[Fréquence] Une fois par mois (selon le permis d'alcool) en dehors des semaines d'examen.
\end{description}

\subsection{Les laits et biscuits du 1er cycle}
\begin{description}
	\item[Description] Une convocation aux membres est envoyée le jour avant pour les convier à se rencontrer au local de l'association. On met à la disposition des membres des verres, environ 2$\times$2L de lait et des boîtes de biscuits.
	\item[Objectif de l'activité] Faire découvrir le local de l'association à de nouvelles personnes étudiantes du DIRO. Offrir un moment pour que les gens du 1er cycle se rencontrent et discutent.
	\item[Responsable suggérés] Représentation de premier cycle.
	\item[Fréquence] Hebdomadaire.
\end{description}

\subsection{La cabane à sucre}
\begin{description}
	\item[Description] Sortie à la cabane à sucre. Le départ et le retour se font par autobus scolaire (1 ou 2).
	\item[Objectif de l'activité]
	\begin{itemize}
		\item[]
		\item Favoriser des rencontres entre les membres. Offrir un moment de socialisation.
		\item Permettre aux personnes étudiantes internationales membres de l'AÉDIROUM de découvrir un aspect de la culture québécoise.
	\end{itemize}
	\item[Responsables suggérés] CVE.
	\item[Fréquence] Une fois par an, au temps des sucres.
\end{description}

\subsection{Les MiDiros}
\begin{description}
	\item[Description] Présentation de sujet divers en informatique.

	Peut autant servir à la présentation de sujet de recherche qu'à la présentation d'outils de développement.

	Habituellement, une conférence d'une heure est donnée par une personne étudiante graduée qui vulgarise et présente son sujet de recherche aux participants.
	\item[Objectif de l'activité]
	\begin{itemize}
		\item[]
		\item Être une plateforme pour présenter des sujets en informatiques hors des cours. Présenter le sujet de recherches de certains étudiants gradués.
		\item Donner un aperçu aux membres des projets de recherche du DIRO.
		\item Informer les personnes étudiantes du 1er cycle sur différents aspects des études supérieures.
	\end{itemize}
	\item[Responsables suggérés] Responsable MiDiro.
 	\item[Fréquence] Une fois aux deux semaines ou en fonction des disponibilités des personnes étudiantes graduées.
\end{description}

\subsection{Le Carnaval de la FAÉCUM}
\begin{description}
	\item[Description] Compétitions multiples s'échelonnant sur près d'un mois à la session d'hiver.
	\item[Objectif de l'activité] Compétitionner à l'échelle de l'université pour affronter d'autres programmes
	\item[Fréquence] Une fois par an, au mois de janvier et février.
	\item[Responsables suggérés] \texttt{null pointer exception}.
\end{description}

\end{document}
